\subsection{Measurement of Time}

This section instructs the reader on how they should think about an inertial frame $S$. Basically, when an inertial frame is mentioned, we consider the measuring of time in frame $S$ by a group of ``magic clock" users where as somebody measures a time $t$, everybody else will have that same measurement.
More specifically,
\begin{quote}
    ``When we speak of an inertial frame S, we will always have in mind a system of axes $Oxyz$ and a team of observes who are stationed at rest throughout $S$ and equipped with synchronized clocks. This allows us to speak of the position $\textbf{r} = (x,y,z)$ and the time $t$ of any event, relative to the frame $S$"
\end{quote}