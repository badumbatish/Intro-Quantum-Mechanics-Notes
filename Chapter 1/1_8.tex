\subsection{The relativity of Time; Time Dilation}
Readers are encouraged to  read the book for the experiment and derivation of the following formulas. In this section, we only mention the formulas, a way to approach a problem and explain the \textbf{proper time} and the \textbf{unique inertia frame}.

\subsubsection{Formulas}

Let $v$ be the speed of a frame $S$, and $c$ be the speed of light. The ratio between $v$ and $c$ is $\beta$
\begin{align*}
    v &= \beta c \\ \\
    \beta &= \frac{v}{c}
\end{align*}

Then the symbol $\gamma$ expressing the ratio between $\Delta t$ and $\Delta t_0$:
\begin{align*}
    \gamma &= \frac{1}{\sqrt{1-\beta^2}} \\\\
     \Delta t &= \gamma \Delta t_0
\end{align*}

\subsubsection{Proper time}

The definition of proper time can sometimes be a little problematic. Suppose we have two frames $A$ and $B$ and an object. Then the proper time would be the time measurement in the frame that two events $x$ and $y$ happen at the same place. 

\begin{ex}
    In the case of a particle, the ``proper" half-life time of a particle is the time when the particle is at rest when it is decaying. The event $x$ can be the time when it starts decaying. The event $y$ can be when it finishes decaying half of its particles.
\end{ex}

\subsubsection{A way to approach problems}
I recommend a way to solve problems involving two frames of reference. Set up a table with three columns, the first one for the measurements' names, the second one is for the normal frame of reference, the third frame is for the unique frame of reference where ``proper time" happens.