\documentclass{article}
\usepackage[utf8]{inputenc}

\usepackage{hyperref} % for referrencing

\usepackage{amsmath}


%%%%%%%%%%%%%%%%%%
% here are more "standard" packages
\usepackage{amssymb} % extra math symbols
\usepackage{amsthm} % fancy theorems & examples
\usepackage{xcolor} % lets me color my comments
%%%%%%%%%%%%%%%%%%%

\newcommand{\jcomm}[1]{{\color{red}(#1)}} % for comments
%%%%%%%%%%%%%%%%%%%%%%%
% this will let us make "enviornments" that look nice, i.e., theorems and examples

% the first curly brace tells you how you will call this in your code, i.e., here I can type \begin{theorem} .... \end{theorem}, or for the 2nd one, \begin{thm} ... \end{thm}
% the second curly brace tells you how it will be dispplayed in the pdf, so the 1st & 2nd will both display as Theorem.
% the square brackets tells you how the numbering works. If it comes at the end, we'll take sub-numbering. So for the 1st, if we are in section 2, the 1st theorem is 2.1, then 2.2, etc.. If the square bracket comes between the curly braces, then we'll use the same numbering. So if I use \begin{lemma} ... \end{lemma} after theorem 2.2, this will be Lemma 2.3
\newtheorem{theorem}{Theorem}[section]
\newtheorem{thm}[theorem]{Theorem}
\newtheorem{lemma}[theorem]{Lemma}
\newtheorem{prop}[theorem]{Proposition}
\newtheorem{cor}[theorem]{Corollary}
\newtheorem{fact}[theorem]{Fact}

% this is a different way of formatting envirnomnets
\theoremstyle{definition}
\newtheorem{defn}[theorem]{Definition}
\newtheorem{remark}[theorem]{Remark}
\newtheorem{rmk}[theorem]{Remark}
\newtheorem{ex}[theorem]{Example}

\title{Quantum Mechanics Notes: Class 250C}
\author{Jasmine Tang }


\begin{document}
\maketitle

\tableofcontents
Use the book of Modern Physics for Scientists and Engineers 2nd Edition
\section{The Space and Time of Relativity}
\subsection{Relativity}
\textbf{Relativity of measurements}: Measurement requires the specification of a reference system relative to which the measurement is to be made.

Einstein's theory of relativity is actually two theories.
\begin{itemize}
    \item Special theory of relativity (excluding unaccelerated frames of reference and gravity): \textbf{Put most effort} into this one
    \item General theory of relativity (including the mentioned things)
\end{itemize}


\subsection{The relativity of Orientation and Origin}

The relativity of Orientation and Origin is about the fact that ??? \jcomm{To be honest I don't even know}

\textbf{Invariant} is the fact that as we move from one reference to another, the laws of motions are unchanging. It works for Newtonian mechanics, electromagnetism, and thermodynamics.

We won't use much \textbf{invariant} in this notes
\subsection{Moving reference frames}

This section is dedicated to saying that ``If Newton's laws are valid in one reference frame, they are also valid in any second frames that moves with \textbf{constant velocity relative to the first}"

The bolden text signifies that if the coordinate system was \textit{accelerating}, Newton's laws falls apart

\begin{rmk}
    In classical physics, the unaccelerated frames in which Newton's laws hold (including the law of inertia??) are often called \textbf{inertial frames}.

    By this remark and by bolden text above, we can say that an accelerated frame is \textbf{noninterntial} because Newton's laws doesn't work in accelerated frames.
\end{rmk}
\subsection{Classical Relativity and the Speed of Light}

Mental experiment in page 7 shows that the \textbf{classical velocity-addition formula} fails to work with the speed of light in a vacumm. Then \textbf{something must be wrong!}

In fact, early scientists of the end of the nineteeth century have an explanation.

\subsubsection{The faulty ether frame}
A faulty earlier explanation: \textbf{The Ether Frame}: The belief that light waves travels through a medium that no one has ever seen or felt. This medium has unusual properties that allows light to travel at the same speed \textit{c} in all directions

The \textbf{ether frame} belief was dispelled by Einstein's relativity.

\subsubsection{Where did we go wrong then?}
We took for granted certain ideas about space and time based on our everyday experiences and saw that Newton's laws is valid in nonaccelerating frames. For accelerating frames, we have ether frame to explain it, which is perfect(!?).

A reader might say: "But, but, but, but it works sooo perfectly, logically and internally, how can we even decide that something this perfect is wrong."

As with any sciences, we rely on experiments to find the truth

\subsubsection{The journey of experimentations}

Assuming the ether frame idea holds true, we set out to measure the speed of light. 

The earlier experiments didn't materialize because we cannot measure small differences in the speed of light detected by direct measurements.

Until, \textbf{Michelson} and \textbf{Morley} comes around with their experience and concludes that light always travels with the same speed in all directions in many different inertial frames, and the notion of a unique ether frame with this property must be abandoned.

\subsection{The Michelson-Morley Experinment}

Ignored due to the ability to omit the section without loss of continuity
\subsection{The postulates of Relativity}

\subsubsection{First posulate of relativity}
\begin{quote}
    If $S$ is an inertial frame and if a second frame $S'$ moves with constant velocity relative to $S$, then $S'$ is also an inertial frame.
\end{quote}

\subsubsection{Second postulate of relativity}
\begin{quote}
    In all inertial frames, light travels through the vacuum with the same speed, $c = 299,792,458 $ m/s in any direction
\end{quote}
\subsection{Measurement of Time}

This section instructs the reader on how they should think about an inertial frame $S$. Basically, when an inertial frame is mentioned, we consider the measuring of time in frame $S$ by a group of ``magic clock" users where as somebody measures a time $t$, everybody else will have that same measurement.
More specifically,
\begin{quote}
    ``When we speak of an inertial frame S, we will always have in mind a system of axes $Oxyz$ and a team of observes who are stationed at rest throughout $S$ and equipped with synchronized clocks. This allows us to speak of the position $\textbf{r} = (x,y,z)$ and the time $t$ of any event, relative to the frame $S$"
\end{quote} 
\subsection{The relativity of Time; Time Dilation}
Readers are encouraged to  read the book for the experiment and derivation of the following formulas. In this section, we only mention the formulas, a way to approach a problem and explain the \textbf{proper time} and the \textbf{unique inertia frame}.

\subsubsection{Formulas}

Let $v$ be the speed of a frame $S$, and $c$ be the speed of light. The ratio between $v$ and $c$ is $\beta$
\begin{align*}
    v &= \beta c \\ \\
    \beta &= \frac{v}{c}
\end{align*}

Then the symbol $\gamma$ expressing the ratio between $\Delta t$ and $\Delta t_0$:
\begin{align*}
    \gamma &= \frac{1}{\sqrt{1-\beta^2}} \\\\
     \Delta t &= \gamma \Delta t_0
\end{align*}

\subsubsection{Proper time}

The definition of proper time can be a little loose 
\subsection{1.11 The Lorentz transformation}

The section starts out with the \textbf{standard configuration}. After that, the incorrect \textbf{Galilean transformation} is introduced as the stepping stone for the correct \textbf{Lorentz transformation} or alternatively the \textbf{Lorentz-Einstein transformation}

\subsubsection{Galilean transformation}

With the \textbf{standard configuration} in page 25

\begin{equation*}
    \begin{aligned}
        x' &= x - vt \\
        y' &= y \\
        z' &= z \\
        t' &= t
    \end{aligned}
\end{equation*}

This set of equations (incorrectly) transforms the coordinates $x,y,z,t$ of any event in $S$ into the coordinates $x', y', z', t'$ of that event in $S'$

\subsubsection{The Lorentz(-Einstein) transformation}

Of course, having studied from chapter 1.1 to chapter 1.10, we are not some fool who thinks the world isn't relativistic anymore. On the contrary, it must be!

Let us identify the ``normal" frame and the ``unique frame".

The unique frame is the frame where the measured object is at rest. Thus, $S'$ is the unique frame and $S$ is the normal frame.

Measuring in $S'$, the distance measured from $O'$ to $P'$ is $x_{S'}' = x'$ but the same distance measure in $S$ is $x_{S}'= x - vt$. Then by the length-contraction formula

\begin{align*}
    x_{S}' &= \frac{x_{S'}'}{\gamma} \\
    x - vt &= \frac{x'}{\gamma} \\
\end{align*}
or 
\begin{align*}
    x' &= \gamma (x - vt)
\end{align*}

We have figured out $x'$ in terms of $x$ and $t$. Since $y' = y$ and $z' = z$, we need to find $t'$ in terms of the variables measured in the frame $S$. 

Following the argument in the book, I will try to manipulate algebraically the equation for $t'$:

\begin{align*}
    t' &= \gamma t - \frac{\gamma^2 - 1}{\gamma v}x \\
    t' &= \gamma (t - \frac{\gamma^2 - 1}{\gamma^2 v}x ) \\
    t' &= \gamma (t - \frac{x}{v} \frac{\gamma ^ 2 -1}{\gamma^2 v}) \\
    t' &= \gamma (t - \frac{x}{v} (1 - \frac{1}{\gamma^2} )) \\
    t' &= \gamma (t - \frac{x}{v} (1 - \frac{1}{\frac{1}{1-\beta^2}} )) \\
    t' &= \gamma (t - \frac{x}{v} (1 - (1-\beta^2) )) \\
    t' &= \gamma (t - \frac{x}{v} \beta^2) \\ 
    t' &= \gamma (t - \frac{x}{v} \frac{v^2}{c^2} )\\
    t' &= \gamma (t - \frac{xv}{c^2})
\end{align*}

Then the set of equations representing the \textbf{Lorentz transformation} is
\begin{align*}
    x' &= \gamma (x - vt) \\
    y' &= y \\
    z' &= z \\
    t' &= \gamma (t - \frac{vx}{c^2})
\end{align*}

Suppose we want to measure something P at rest in $S$, then as seen from point P, the car will be moving forward with velocity $v$ instead of seeing from point $P'$ the whole frame $S$ moves back $-v$ velocity. Then we can apply the Lorentz transformation by replacing $x', y', z', t'$ with $x, y, z, t$ to form  the \textbf{inverse Lorentz transformation}

\begin{align*}
    x &= \gamma (x' + vt') \\
    y &= y' \\
    z &= z' \\
    t &= \gamma (t' + \frac{vx'}{c^2})
\end{align*}


\end{document}
