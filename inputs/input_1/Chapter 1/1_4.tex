\subsection{Classical Relativity and the Speed of Light}

Mental experiment in page 7 shows that the \textbf{classical velocity-addition formula} fails to work with the speed of light in a vacumm. Then \textbf{something must be wrong!}

In fact, early scientists of the end of the nineteeth century have an explanation.

\subsubsection{The faulty ether frame}
A faulty earlier explanation: \textbf{The Ether Frame}: The belief that light waves travels through a medium that no one has ever seen or felt. This medium has unusual properties that allows light to travel at the same speed \textit{c} in all directions

The \textbf{ether frame} belief was dispelled by Einstein's relativity.

\subsubsection{Where did we go wrong then?}
We took for granted certain ideas about space and time based on our everyday experiences and saw that Newton's laws is valid in nonaccelerating frames. For accelerating frames, we have ether frame to explain it, which is perfect(!?).

A reader might say: "But, but, but, but it works sooo perfectly, logically and internally, how can we even decide that something this perfect is wrong."

As with any sciences, we rely on experiments to find the truth

\subsubsection{The journey of experimentations}

Assuming the ether frame idea holds true, we set out to measure the speed of light. 

The earlier experiments didn't materialize because we cannot measure small differences in the speed of light detected by direct measurements.

Until, \textbf{Michelson} and \textbf{Morley} comes around with their experience and concludes that light always travels with the same speed in all directions in many different inertial frames, and the notion of a unique ether frame with this property must be abandoned.
